%%
% This is an Overleaf template for scientific articles and reports
% using the TUM Corporate Desing https://www.tum.de/cd
%
% For further details on how to use the template, take a look at our
% GitLab repository and browse through our test documents
% https://gitlab.lrz.de/latex4ei/tum-templates.
%
% The tumarticle class is based on the KOMA-Script class scrartcl.
% If you need further customization please consult the KOMA-Script guide
% https://ctan.org/pkg/koma-script.
% Additional class options are passed down to the base class.
%
% If you encounter any bugs or undesired behaviour, please raise an issue
% in our GitLab repository
% https://gitlab.lrz.de/latex4ei/tum-templates/issues
% and provide a description and minimal working example of your problem.
%%


\documentclass[
  english,        % define the document language (english, german)
  font=times,     % define main text font (helvet, times, palatino, libertine)
  twocolumn,      % use onecolumn or twocolumn layout
]{tumarticle}


% load additional packages
\usepackage{lipsum}


% article metadata
\title{LLMs for Structured Constraint Generation}

\author[email=jaden.rotter@tum.de]{Jaden Rotter}
\author[email=saumil.savani@tum.de]{Saumil Savani}

\date{July 8, 2025}


\begin{document}

\maketitle

\begin{abstract}
Abstract
\end{abstract}

\section{Introduction}
As the internet grows exponentially, guaranteeing the reliability and consistency of web data for use in applications has become increasingly important. 
However, the sheer volume of data makes manual validation infeasible, driving the need for automated, machine oriented solutions. 
The Semantic Web, an extension of the internet, addresses this by enabling machines to read and understand data through structured languages defined by standard ontologies. 
On this structured data, further languages such as Shapes Constraint Language (SHACL) allow parameters and constraints to be defined in order to validate the underlying data. 
Unfortunately, as both the volume and complexity of data increase, validation requirements become more sophisticated, making the design of structured validation rules increasingly challenging.
To simplify human validation tasks, we propose a solution involving large language models (LLMs) to help convert natural language (NL) into structured SHACL shapes. 


Our solution enables users to construct natural language prompts, which are then passed to a fine-tuned model that generates the corresponding SHACL shapes, significantly simplifying the process of creating these complex, syntax-heavy specifications. 
To train the model to recognize structured SHACL patterns, we developed a data generation pipeline that produces pairs of natural language descriptions and their SHACL equivalents. 
Using this synthetic dataset, we fine-tuned several open-source models from Hugging Face and validated their outputs against a predefined ground truth.
For evaluation, we implemented an automated validation pipeline that applies natural language processing (NLP) techniques to assess both syntactic and semantic similarity to the ground truth. 
Additionally, we performed manual evaluation to ensure accuracy and completeness.


\section{Background}
In similar structured language generation work, LLM have shown success 

\subsection{First subsection}
\lipsum[2]
\subsection{Second subsection}
\lipsum[3]

\section{Methods}
\lipsum[4-5]

\section{Outlook}
\lipsum[6]

\end{document}
